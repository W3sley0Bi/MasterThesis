\chapter{Conclusion\label{cha:chapter6}}
This final chapter describes the work that has been done throughout the course of this thesis. It revises the main ideas behind the application developemt and summarizing the key steps taken during the implementation. 
It then highlights some of the challenges that has been addressed and problems that still remain unsolved.
The chapter ends with a brief outlook on how the current solution could be improved and extended in the future.
\section{Summary\label{sec:summary}}
% \noindent The work done can be summarized into the following work steps

% \begin{itemize}
% 		\item Analysis of available technologies for RDF data generation and interactive graph representation
% 		\vspace{-0.11in} 
% 		\item Selection of 3 relevant services for implementation
% 		\vspace{-0.11in} 
% 		\item Design and implementation of X on Windows
% 		\vspace{-0.11in} 
% 		\item Design and implementation of X on mobile devices
% 		\vspace{-0.11in} 
% 		\item Documentation based on X
% 		\vspace{-0.11in} 
% 		\item Evaluation of the proposed solution
% \end{itemize}

The first step that has been done for this project is a research and ananysis of RDFs and the technologies that are currently orbiting around like the semantic web or the owl ontology language.

After ahving new know, a a review on the programs and applications that are made for the rdf generation or visualization like protege and the GAIA project. 



\section{Dissemination\label{sec:dissemination}}

Who uses your component or who will use it? Industry projects, EU projects, open source...? Is it integrated into a larger environment? Did you publish any papers?

\section{Problems Encountered\label{sec:problems}}

Summarize the main problems. How did you solve them? Why didn't you solve them?

\section{Outlook\label{sec:outlook}}

Future work will enhance Component X with new services and features that can be used ...